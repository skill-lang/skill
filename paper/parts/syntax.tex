\section{Syntax}

We use the tokens \verb/<id>/, \verb/<string>/, \verb/<int>/ and \verb/<comment>/. They equal C-style identifiers, strings, integer literals and comments respectively. We use a comment token, because we want to emit the comments in the generated code, in order to integrate nicely into the target languages documentation system.

\subsection{The Grammar}
The grammar of a \gls{skill} definition file is defined as:
\begin{verbatim}
UNIT :=
  INCLUDE*
  DECLARATION*

INCLUDE := 
  ("include"|"with") <string> ";"?

DECLARATION :=
  DESCRIPTION
  <id>
  ((":"|"with"|"extends") <id>)?
  "{" FIELD* "}"
  
FIELD :=
  DESCRIPTION
  (CONSTANT|DATA) ";"?
  
DESCRIPTION := 
  (RESTRICTION|HINT)*
  <comment>?
  (RESTRICTION|HINT)*
  
RESTRICTION :=
  "@" <id> ("(" (R_ARG ("," R_ARG)*)? ")")? ";"?
  
R_ARG := ("%"|<int>|<string>)

HINT := "!" <id> ";"?
  
CONSTANT :=
  "const" TYPE <id> "=" <int>
  
DATA :=
  "auto"? TYPE <id>
  
TYPE :=
  ("map" MAPTYPE
  |"set" SETTYPE
  |"list" LISTTYPE
  |ARRAYTYPE)
  
MAPTYPE :=
  "<" GROUNDTYPE ("," GROUNDTYPE)+ ">"
  
SETTYPE :=
  "<" GROUNDTYPE ">"
  
LISTTYPE :=
  "<" GROUNDTYPE ">"
  
ARRAYTYPE :=
  GROUNDTYPE
  ("[" (<id>|<int>)? "]")?
  
GROUNDTYPE :=
  (<id>|"annotation")

\end{verbatim}
Note: The Grammar is LL(1).\footnote{In fact it can be expressed as a single regular expression.}

Comment: The optional \texttt{;} at the end of includes or definitions are for convenience only.

\subsection{Reserved Words}

The language itself has only the reserved words \textbf{annotation}, \textbf{auto}, \textbf{const}, \textbf{include}, \textbf{with}, \textbf{bool}, \textbf{map}, \textbf{list} and \textbf{set}.

However, it is strongly advised against using any identifiers which form reserved words in a potential target language, such as Ada, C++, C\#, Java, JavaScript or Python. A complete list is given in appendix \ref{app:keywords}.

\subsection{Examples}

\begin{lstlisting}[label=blockExample,caption=Running Example,language=skill]
/** A source code location. */
SLoc {
  i16 line;
  i16 column;
  string path;
}

Block {
  SLoc begin;
  SLoc end;
  string image;
}

IfBlock : Block {
  Block thenBlock;
}

ITEBlock : IfBlock {
  Block elseBlock;
}
\end{lstlisting}

\subsubsection*{Includes, self references}

\begin{lstlisting}[label=example2a,caption=Example 2a,language=skill]
with "example2b.skill"

A {
  A a;
  B b;
}
\end{lstlisting}

\begin{lstlisting}[label=example2b,caption=Example 2b,language=skill]
with "example2a.skill"

B {
  A a;
}
\end{lstlisting}

\subsubsection*{Unicode}
The usage of non ASCII characters is completely legal, but discouraged.
\begin{lstlisting}[label=unicode,caption=Unicode Support,language=skill]
/** some arguably legal unicode characters. */
ö {
  ö ∀;
  ö €;
}
\end{lstlisting}
